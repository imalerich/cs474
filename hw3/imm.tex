\documentclass{jhwhw}
\author{Ian Malerich}
\title{Com S 474: Homework 3}

\usepackage{amssymb, amsfonts, mathtools, graphicx, breqn, minted, graphicx, subfig, float}
\usemintedstyle{friendly}

\DeclareMathOperator*{\argmax}{arg\,max}

\begin{document}

\raggedright

%% Problem 1
\problem{}

\begin{enumerate}
    \item Find the equation of the hyperplane (in terms of $w$) WITHOUT solving a quadratic programming problem.
	Make a sketch of the problem.
    \item Calculate the margin.
    \item Find the $\alpha$'s of the SVM for classification.
    \item Perform SVM with Matlab or R with RBF kernel and plot the result. Is this result different
	from the one you obtained by hand?
\end{enumerate}

\solution

\part
    \begin{center}
	\begin{figure}[H]
	    \centering
	    \subfloat[Original Data]{{ \includegraphics[scale=0.45]{data} }}
	    \subfloat[Transformed Data]{{ \includegraphics[scale=0.45]{transformed} }}
	\end{figure}
    \end{center}

    From this transformed data, we can clearly see our two support vectors (one for each class).
    For the orange class (-1) this is at point [1,1] and for the blue class (+1) this is at point [2,2].
    SVM requires that all points be on their respective "positive" sides of the hyperplane, thus we can
    see that this hyperplane must pass between the two support vectors we have chosen.
    This means, that if we imagine a line segment connecting connecting the points \{[1,1], [2,2]\} our
    hyperplane MUST intersect that line segment. Further, as these points are support vectors, they must lie
    at equal distance to the hyperplane, from these observations we conclude that the hyperplane will intersect
    these two points at their midpoint, this will be the point $\frac{1}{2}[1 + 2, 1 + 2] = [\frac{3}{2}, \frac{3}{2}]$. \\

    As our transformed data exists in two dimensions, our hyperplane will take the form of a line, as we know an
    intercept of this line, the only thing that remains to be found is the slope of the line, once we have that we can
    derive an equation for this line in the desired form. \\

    Intuitively we would expect the hyperplane to be perpendicular to our imaginary line segment connecting our
    two support vectors. To see why this must be the case, simply consider the case where it is not.
    At one extreme, our hyperplane is in line with both support vectors, producing a margin equal to 0, as we rotate 
    our line, this margin will increase in size. It will continue to increase in size until it reaches the perpendicular
    we are considering, if we continue to rotate beyond the perpendicular, each edge will begin to approach
    the opposite class, thus resulting in a decrease again in our margin. Thus maximum margin is achieved at the 
    perpendicular. As the line segment through \{[1,1],[2,2]\} has slope 1, this implies our hyperplane has slope -1.

    In standard form this means our line will take the form: \\
    $$
	ax + b = -1x + 3
    $$
    Some simple algebra will produce the vectorized form: \\
    \begin{align*}
	-x + 3 &= y &\\
	x + y - 3 &= 0&\\
	[1; 1]^T[x; y] - 3 &= 0 &\\
    \end{align*}
    From this we have w = [1, 1] and b = -3. \\

    To check our work we need to consider our class labels (else we would simply multiply both sides by -1),
    we expect $w'[2;2]+b = 1$ and $w'[1;1]+b = -1$. Plugging these values into matlab produces the expected results
    so I'll assume I did my work correctly.

    \begin{center}
	\includegraphics[scale=0.8]{boundary}
    \end{center}

    \clearpage

\part

    The margin will simply be the distance between my two support vectors.

    \begin{align*}
	dist([1, 1], [2, 2]) &= sqrt((2 - 1)^2 + (2 - 1)^2) &\\
	&= sqrt(2(1)^2) &\\
	&= sqrt(2) &\\
    \end{align*}

    Thus we find we have a margin of $sqrt(2)$.

\part

    Note that the majority of our points are not support vectors, for all of these points we have that 
    $\alpha = 0$. Note that our support vectors are observation 1 and 5, thus we still need to find
    $\alpha_1$ and $\alpha_5$. 
    From my notes we have that 
    $ w^\intercal = \Sigma_{i=1}^{m}\alpha_i Y_i x_i^\intercal$, noting that each class only has 
    one support vector we can simplify this as follows
    $$
	w^\intercal = \alpha_1 * [2; 2]^\intercal + -\alpha_5 * [1; 1]^\intercal
    $$
    This is a 2x2 linear system, format it as such.
    $$
	\begin{bmatrix}
	    1 \\ 1
	\end{bmatrix}
	= 
	\begin{bmatrix}
	    2 & -1 \\
	    2 & -1 \\
	\end{bmatrix}
	\begin{bmatrix}
	    \alpha_1 \\
	    \alpha_5
	\end{bmatrix}
    $$
    From here it is pretty obvious that $\alpha_5 = \alpha_1$ denote this $\delta$
    and we have $1 = 2\delta - 1\delta = (2 - 1)\delta = \delta$.
    Thus $\delta = 1$.

    \begin{align*}
	\alpha_1 &= 1 &\\
	\alpha_5 &= 1 &\\
	\alpha_i &= 0 $ otherwise$ &\\
    \end{align*}

\part

    \inputminted[frame=lines,framesep=2mm]{matlab}{p1.m}

%% Problem 2
\problem{}

    Verify by means of simulation that each bootstrap sample will contain 1 - 1/e $\approx$ 63.2 of the original sample.

\solution

    \begin{center}
	\includegraphics[scale=0.8]{p2.png}
    \end{center}
    \clearpage
    \inputminted[frame=lines,framesep=2mm]{matlab}{p2.m}

%% Problem 3
\problem{}

    Compare SVM and LS-SVM on the ozon level detection data set on Blackboard.
    This dataset contains 72 measurement variables and was measured between 1/1/1998 and 12/31/2004.
    All missing values have been removed. The goal is to detect whether there was too much ozon (class label 0)
    or a normal day (class label 1). The class label is the last variable. Set up the simulation and clearly describe what
    you are doing and why. Finally, state, according to your findings (boxplots, ROC curves, etc.), the best
    classifier for this problem.

\solution

    \begin{center}
	\includegraphics[scale=0.45]{p3_boxplot.png}
    \end{center}

    \inputminted[frame=lines,framesep=2mm]{matlab}{p3.m}

\end{document}
