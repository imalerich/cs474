\documentclass{jhwhw}
\author{Ian Malerich}
\title{Com S 474: Homework 3}

\usepackage{amssymb, amsfonts, mathtools, graphicx, breqn, minted}
\usemintedstyle{friendly}

\DeclareMathOperator*{\argmax}{arg\,max}

\begin{document}

%% Problem 1
\problem{}

\begin{enumerate}
    \item Find the equation of the hyperplane (in terms of $w$) WITHOUT solving a quadratic programming problem.
	Make a sketch of the problem.
    \item Calculate the margin.
    \item Find the $\alpha$'s of the SVM for classification.
    \item Perform SVM with Matlab or R with RBF kernel and plot the result. Is this result different
	from the one you obtained by hand?
\end{enumerate}

\solution

\part TODO
\part TODO
\part TODO
\part TODO

%% Problem 2
\problem{}

    Verify by means of simulation that each bootstrap sample will contain 1 - 1/e $\approx$ 63.2 of the original sample.

\solution

    TODO

%% Problem 3
\problem{}

    Compare SVM and LS-SVM on the ozon level detection data set on Blackboard.
    This dataset contains 72 measurement variables and was measured between 1/1/1998 and 12/31/2004.
    All missing values have been removed. The goal is to detect whether there was too much ozon (class label 0)
    or a normal day (class label 1). The class label is the last variable. Set up the simulation and clearly describe what
    you are doing and why. Finally, state, according to your findings (boxplots, ROC curves, etc.), the best
    classifier for this problem.

\solution

    TODO

\end{document}
